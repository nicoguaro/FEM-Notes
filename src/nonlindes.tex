\chapter[Nonlinear Equilibrium in FEM]{Nonlinear Equilibrium Equations in a Finite Element Discretization}

\section{General Equilibrium Equations}
In this section we develop the incremental equations needed in a general non-linear problem.  First, geometric non-linearities valid for any stress-strain pair are considered.  Second, focus is shifted towards the particular case of the Second Piola-Kirchoff-Green Lagrange Strain pair.

Let   represent the current (deformed) configuration of a deformable medium with equilibrium equations and boundary conditions

\begin{equation}
{\sigma _{ij,j}} + {f_i} = 0	 
\end{equation}			

\begin{equation}
{t_i} = {\sigma _{ij}}{\hat n_j}
\end{equation}	 		

where   represents the Cauchy definition of stress (or force per unit of deformed surface) and   the corresponding tractions vector at a surface with outward normal  .  In the equilibrium equations stated in (1) the domain   is unknown which makes the problem inherently non-linear.  On the other hand this equilibrium statement is mathematically indeterminate since 9 unknown stress components must be solved out of 6 field equations.  The indeterminacy is destroyed after the problem is kinematically described and connected to the stress field via constitutive modeling which can involve additional sources of non-linearity.  To summarize the problem is non-linear since the domain   is unknown, this is what we typically call a Geometric Non-linearity and will be reflected in the mathematical description of changes in configuration.



\subsection{Lagrangian description of the equilibrium equations}
The conceptual definition of Cauchy stress is the only one useful for the engineer since it describes forces per unit deformed surface.  This validity is clearly identified in the fact that the equilibrium equations stated in (1) have been formulated in the stressed deformed domain   where the body is in fact in equilibrium.  The difficulty associated with the unknown domain may be dealt with in alternative ways.  In the case of a solid body it is convenient to refer everything to the undeformed or reference configuration (with known domain)   and proceed from there using linearization which corresponds to a Total Lagrangian (TL) approach.  It is then useful to understand the transformation of the problem to   via pullback operations.  This process implies the introduction or consideration of mathematically  defined stress definitions and newly developed kinematic descriptions.  In this section we consider such transformations first from a dynamic point of view and later using thermodynamic principles we address the problem of identifying the proper strain measures.
\begin{figure}[h]
\centering
\includegraphics[width=8cm]{img/figure1_2.pdf}
\caption{Definition of the natural domain}
\label{fig:natural large domain}
\end{figure}

\subsubsection*{Nanson's formula}
For the treatment that follows it will result useful to consider the relation between oriented differential surface elements in the reference and deformed configurations in the so-called Nanson's formula;

\begin{equation}
\hat{n}_{i}dS=\frac{\rho}{\rho _0}\hat{N}_I f_{Ii}dS_0
\label{nanson}
\end{equation}

\subsubsection*{Lagrangian stress definition}
Let be the differential force associated to the Cauchy physical stress such

\begin{equation}
d{P_i} = {t_i}dS \equiv {\sigma _{ji}}{n_j}dS
\label{diff stress 1}
\end{equation}

Assume that   can also be obtained in the undeformed configuration associated with a new traction definition;

\begin{equation}
d{P_i} = t_i^0d{S_0} \equiv T_{Ii}^0{N_I}d{S_0} \equiv {\sigma _{ji}}{n_j}dS
\label{diff stress 2}	
\end{equation}

Using Nanson's formula it is possible to write;

\begin{equation}
T_{Ii}^0{N_I}d{S_0} \equiv {\sigma _{ji}}\frac{{{\rho _0}}}{\rho }{N_I}{f_{Ij}}d{S_0}
\label{equiv}
\end{equation}

Which simplifies into

\begin{equation}
T_{Ii}^0 = \frac{{{\rho _0}}}{\rho }{f_{Ij}}{\sigma _{ji}}
\label{simplified}
\end{equation}

and this is the asymmetric First Piola-Kirchoff stress tensor.

Assume now that there is a pseudo-force in the undeformed configuration   which results from a pullback operation on the physical force  .  Recalling the kinematic connection between the current and undeformed configurations

\begin{equation}
d{X_I} = {f_{Ii}}d{x_i}
\label{pseudo}
\end{equation}

where   is the inverse deformation gradient, we can write for the force and pseudo-force vectors;

\begin{equation}
d{\tilde P_I} = {f_{Ii}}d{P_i}.
\label{for and psefor}
\end{equation}

Now we can define a tractions vector   in the undeformed configuration and associated with the pseudo-force   like

\begin{equation}
d{\tilde P_I} = {\tilde t_I}d{S_0} \equiv {T_{JI}}{N_J}d{S_0}
\label{unde tracts 1}
\end{equation}

where we have at the same time introduced the associated stress tensor.  It then directly follows that;

\begin{equation}
{T_{JI}}{N_J}d{S_0} = {f_{Ii}}d{P_i} \equiv {f_{Ii}}{\sigma _{ji}}{n_j}dS
\label{unde tracts 2}	 
\end{equation}

once again using Nanson's formula we can write

\begin{equation}
{T_{JI}}{N_J}d{S_0} = {f_{Ii}}{\sigma _{ji}}\frac{{{\rho _0}}}{\rho }{N_J}{f_{Jj}}d{S_0}
\label{unde tracts 3}
\end{equation}

and

\begin{equation}
{T_{JI}} = \frac{{{\rho _0}}}{\rho }{f_{Jj}}{\sigma _{ji}}{f_{Ii}}
\label{unde tracts 4}
\end{equation}

which corresponds to the symmetric Second Piola-Kirchoff stress tensor.

For the derivations that follow and in the actual computational implementation it will be convenient to have the inverse relationships expressing the Cauchy stress tensor in terms of the First and Second Piola-Kirchoff stress definitions.

Once again recall that   and use the definition found for the first PK stress tensor to write







\section{Non-linear problems}
In this section we will study the basic 1-single independent variable problem of numerically finding possible roots to the non-linear equation;

\begin{equation}
f(x) = 0
\label{1D}
\end{equation}

and then we will proceed via generalization to the more realistic case of a system of non-linear equations of the form (29).  This last general form will then be converted to the particular case of an equilibrated system of forces as encountered in the Finite Element Method or other numerical techniques appearing in the solution of continuum mechanics problems.

Since the problem is assumed non-linear the root finding process must use iterations.  The general idea is that of establishing an initial trial solution and then proposed an efficient algorithm to improve the solution until some desired tolerance is satisfied.

In this section we will concentrate on Newton methods since most of the followed techniques in FEM analysis are either Newton or improved Newton Methods.  A general and detailed survey of root finding and non-linear equation solving can be found in Press et al.

The basic idea of the Newton method is that of taking an initial guess ${}_iX$ (the trial solution), extending the tangent line at this point until it crosses zero and then taking as the next approximation ${}_{i + 1}X$ the abscissa of that zero crossing, see \cref{fig:1Dnewton}.

\begin{figure}[h]
\centering
\includegraphics[width=9cm]{img/newton1d.pdf}
\caption{Fundamental Newton-Raphson iteration}
\label{fig:1Dnewton}
\end{figure}

The main disadvantage of the Newton method is that we formally need to obtain the function derivative $Y_{i}^{'}$ at every point.

Let $\delta$ be the correction to an approximation $X$ to the actual root $P$ such;

\begin{equation}
P = X + \delta
\label{correction}
\end{equation}

Using a Taylor series expansion for $f(P)$ in the neighborhood of $X$ we have;

\begin{equation}
\begin{aligned}
f(P) & \equiv f(X + \delta )=f(X)+{f^{'}}(X)+\frac{{{f^{{''}}}(X)}}{2}{\delta ^2}\\
f(P) & \equiv f(X + \delta )={f^{'}}(X)\delta  + \frac{{{f^{{''}}}(X)}}{2}{\delta ^2}\\
f(P) & \equiv f(X + \delta )=f(X) + {f^{'}}(X)(X - P) + \frac{{{f^{{''}}}(X)}}{2}{(X - P)^2} 
\label{T1}
\end{aligned}
\end{equation}

Using $f(P) = 0$ and linearizing the Taylor series yields;

\begin{equation}
P \approx X - \frac{{f(X)}}{{{f^{'}}(X)}}
\label{T2}
\end{equation}

From \cref{T1} it is clear that the correction $\delta$ satisfies;

\[\delta  =  - \frac{{f(X)}}{{{f^{'}}(X)}}\]

and we can write the general Newton-Raphson algorithm like;

\begin{equation}
\begin{aligned}
{}_{i + 1}X &= {}_iX + {}_i\delta \\
{}_i\delta & =  - \frac{{f({}_iX)}}{{f'({}_{i + 1}X)}}
\label{T3}
\end{aligned}
\end{equation}

In an actual finite element algorithm we are to find an incrementally related history of discrete roots ${N_{roots}}$ such

\begin{equation}
{}^{I + 1}X = {}^IX + {}^{I + 1}\Delta X
\label{IN-1}
\end{equation}

and where

\begin{equation}
\begin{aligned}
f({}^{I + 1}X)& = 0\\
f({}^IX)& = 0
\label{R1}
\end{aligned}
\end{equation}
