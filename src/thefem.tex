\chapter{The Finite Element Method}

\section{Brief review of the linearized theory of elasticity model}
Here we present a brief description of the boundary value problem governing the response of an elastic body. For a full discussion of the model and its mathematical aspects the reader is referred to \cite{shames1997elastic}.

The governing equations (in terms of stresses) stem from the principle of conservation of linear momentum and conservation of moment of linear momentum. The former leads to a set of 3 partial differential equations in the components of the stress tensor while the latter leads to the symmetrie in the stress tensor.

\begin{equation} \label{eq:pde}
\begin{aligned}
&\sigma_{ij,j} + {f_i} = \rho\ddot{u}_i \quad \forall\ \vb{x} \in V,\, t \in \mathbb{R}^{+}\\
&\sigma_{ij}=\sigma _{ji}.
\end{aligned} 
\end{equation}

In \cref{eq:pde} $\sigma_{ij}$ is the stress tensor; $f_i$ is the vector of body forces; and $u_i$ is the displacements vector.

Denoting the tractions vector associated with a surface with normal direction $\hat{n}_{j}$ by $t_i^{\hat n}$ we have the complete BVP as follows:

\begin{equation} \label{eq:bcs}
t_i^{\hat n} = \sigma_{ij} \hat{n}_{j} \quad \forall \in \vb{x} \in S.
\end{equation}

\Cref{eq:pde} correspond to 6 equations with 12 unknowns (the 9 components of the stress tensor and the 3 components of the displacements vector ) and the system is undetermined. In order to have a solvable BVP we must introduce kinematic strain-displacement relations and a stress-strain law. In the case of infinitesimal theory of elasticity the strain-displacement relation is given by:

\begin{equation}\label{eq:kin}
\varepsilon_{ij} = \frac{1}{2}(u_{i,j} + u_{j,i})
\end{equation}

where the term $\epsilon_{ij}$ is the symmetric component of the displacements gradient tensor. The components of the strain tensor describe the distrotions and changes in magnitude (volumetric changes) of the material point in the continuum model. Now the simplest stress-strain (constituive) relationship is given by Hooke's law\footnote{Despite the name of \emph{law} used, this relation is not always valid, but is a good approximation for small strains.}

\begin{equation} \label{eq:Hooke}
\sigma_{ij} = 2\mu \varepsilon_{ij} + \lambda \varepsilon_{kk}\delta_{ij} \enspace .
\end{equation}

where $\mu$ and $\lambda$ are material constants. The problem involves now a total of 18 equations and 18 unknowns can be solved if subjected to properly specified boudnary conditions.

\subsection*{Displacement formulation}
Substituting \cref{eq:kin} in \cref{eq:Hooke} and the result in \cref{eq:pde} yields after some manipulation:

\begin{equation} \label{eq:navier}
(\lambda  + \mu)u_{j,ij} + \mu u_{i,jj} + {f_i} = \rho \ddot{u}_i \quad \forall \vb{x} \in V,\, t \in \mathbb{R}^{+}.
\end{equation}

Since \cref{eq:navier} ( simultaneously describing equilibrium, kinematic relations and constituive response) is a second order equation governing the displacemt field possible boundary conditions are in terms of the variable itself or its first oder derivatives. For a well-possed problem the following is a set of valid boundary conditions: 

\begin{equation} \label{eq:Wellbcs}
\begin{split}
&t_i^{\hat n} = \sigma _{ij} \hat n_{ij} \quad \forall\ \vb{x} \in S_t\\
& {u_i} = \bar{u}_i \quad \forall \vb x \in S_u
\end{split}
\end{equation}

and where ${S_t} \cup {S_u} = S$ and ${S_t} \cap {S_u} = \emptyset $. 


In the particular case in which $u_i$ is not a function of time, we obtain the static version of the BVP, i.e,
\begin{equation}
\begin{split}
&\left(\lambda  + \mu \right)u_{j,ij} + \mu u_{i,jj} + {f_i} = 0 \quad \forall \vb{x} \in V \\
&t_i^{\hat n} = \sigma _{ij} \hat n_{ij} \quad \forall\ \vb{x} \in S_t\\
& {u_i} = \bar{u}_i \quad \forall \vb x \in S_u
\end{split}
\end{equation}

Notice that the tractions BC 
\[t_i^{\hat n} = \mu (u_{i,j} + u_{j,i}) \hat{n}_j + \lambda u_{k,k} \delta_{ij}\hat{n}_j \enspace ,\]

actually involves first order displacemts derivative and as such it is a Neumann boundary condition on $u_i$.
%

\subsection{Equivalence between strong and weak forms}
\subsubsection{Strong form}
The strong form corresponds to the differential formulation of the problem, it is denoted by $\{ S \}$ and it reads:

Given $f_i$, $t_i^{\hat n}$ and ${\bar u_i}$ find ${u_i}:V \to \mathbb{R}$ such:
%
\begin{equation} \label{eq:navier_2}
\begin{split}
&(\lambda  + \mu)u_{j,ij} + \mu u_{i,jj} + f_i = 0 \quad \forall \vb{x} \in V \\
&t_i^{\hat n} = \sigma _{ij} \hat{n}_{ij} \quad \forall \vb{x} \in S_t\\
&u_i = \bar{u}_i \quad \forall \vb{x} \in S_u
\end{split}
\end{equation}

In \cref{eq:navier_2} the boundary conditions specified by the traction vector $t_i^{\hat n}$ correspond to the natural boundary conditions, while those specified in terms of the displacements vector $\bar u_i$ represent the essential boundary conditions.

\begin{itemize}
\item We are interested in developing methods to obtain approximate solutions to $\{S\}$.
\item The FEM is formulated starting from a statement equivalent to $\{ S \}$ in which we use trial functions until certain prescribed conditions are met.
\item We will look for solutions $u_i$ subject to the following conditions:
\begin{align*}
&u_i = \bar u_i\\
&\intL_S \left(\pdv{u_i}{x_j} \right)^2 dS < \infty
\end{align*}

\end{itemize}

The first condition corresponds to the satisfaction of the essential boundary condition, while the second corresponds to the functions being square integrable. The space of functions satisfying the above two conditions is denoted by $\varsigma$ and formally defined like
%
\[\varsigma = \left\{u_i\left| {u_i} \in H, {u_i} = \bar{u}_i \in S_u \right. \right\} \enspace .\]
%
On the other hand, in order to validate the introduced trial functions we also need testing functions $w_i$ also called in the FEM literature weighting or distribution functions. These functions are arbitrary apart from having to satisfy the following conditions:
%
\begin{align*}
&w_i = 0 \quad in \quad {S_u}\\
&\intL_S \left(\pdv{w_i}{x_j}\right)^2 dS < \infty
\end{align*}
%
In what follows we formally denote the space of these functions by $V$ and define it like
%
\[V = \left\{ w_i\left| w_i \in H, u_i = w_i=0 \in S_u \right. \right\} \enspace .\]

\subsubsection{Weak form}
Here we will show that the equilibrium statement represented in the differential formultaionthe problem can be described in an alternative forms. In such description the continuity requirement for the trial functions is weaker than in the strong form leading to the term "weak" statement. Here this alternative representation will be denoted like $\{W\}$ and it reads;

Given $f_i$, $t_i^{\hat n}$ and ${\bar u_i}$ find ${u_i}:V \to \mathbb{R}$ and $\forall {w_i} \in V$ such:

\[\intL_V \sigma _{ij,j} w_{i,j}\, dV - \intL_V f_i w_i\, dV  - \intL_{S_t} t_i^{\hat n} w_i\, dS = 0\]

\subsubsection*{Proof 1:}
Let $u_i \in \varsigma $ be a solution to $\{S\}$ and let $w_i \in V $. Forming the inner product of the equilibrium statement given in \cref{eq:pde} with $w_i$ and forcing the integral over the domain to be zero we have
%
\[\intL_V (\sigma_{ij,j} + f_i ){w_i}\, dV = 0 \enspace ,\]
%
expanding the terms in the integrand and integrating by parts the first term on the left we have
%
\[\intL_V \sigma _{ij,j} w_i\, dV + \intL_V f_i w_i\, dV = 0 \enspace .\]
\[ - \intL_V w_{i,j} \sigma _{ij}\, dV  + \intL_S \sigma _{ij} \hat{n}_j w_i\, dS  + \intL\limits_V w_i f_i\, dV = 0 \]

since $w_i \in V$ it follows that $w_i = 0$ in $S_u$ from which

\begin{equation}\label{weak}
\intL_V \sigma _{ij,j} w_{i,j}\, dV - \intL_V f_i w_i\, dV  - \intL_{S_t} t_i^{\hat n} w_i\, dS = 0
\end{equation}

Now, considering that $u_i$ is solution of the strong form $\{S\}$ it must satisfy $u_i = \bar u_{i} \quad \in \quad S_u$ and as a result $u_i \in \varsigma$. On the other hand, since $u_i$ satisfies \cref{weak} $\forall {w_i} \in V$ we have that $u_i$ satisfies the definition of weak solution specified in $\{ W \}$.

\subsubsection*{Proof 2:}
Let $u_i$ be a solution of $\{W\}$ and thus $u_i \in \varsigma$ which means that
\[u_i = \bar u_{i} \quad \in \quad S_u\]
and that it satisfies
\[\intL_V \sigma _{ij} w_{i,j}\, dV - \intL_V f_i w_i\, dV - \intL_{S_t} t_i^n w_i\, dS = 0 \enspace ,\]
integrating by parts,
\[-\intL_V \sigma_{ij,j} w_idV + \intL_S \sigma_{ij} n_j w_i dS  - \intL_V f_i w_i dV - \intL_{S_t} {t_i^n} w_i dS = 0\]

Since ${w_i} \in V$ we have that ${w_i}=0$ in $S_u$ and therefore
\[\intL_V w_i(\sigma_{ij,j} + f_i)dV + \intL_{S_t} w_i( \sigma_{ij} n_j - t_i^n )dS = 0 \]
from which
\begin{equation} \label{equil_2}
\begin{split}
&\sigma_{ij,j} + f_i = 0 \quad \vb{x} \in V \\
&t_i^n = \sigma_{ij} n_j \quad \forall \vb{x} \in S_t\\
&{u_i} = \bar{u}_i \quad \forall \vb{x} \in S_u
\end{split}
\end{equation}

which is once again the strong form of the problem given in \cref{eq:pde}.

\subsection*{Simple wedge under self-equilibrated loads}
Consider the double wedge of side $\ell$ and internal angle $2 \phi$ shown in \cref{fig:WEDGE}. It is assumed to be contained in the $X-Y$ plane, with loading conditions satisfying a plane strain (or plane stress) idealization. The material is elastic with Lame constants $\lambda$ and $\mu$. The wedge is loaded by uniform tractions of intensity $S$ applied over its four faces in such a way that the wedge is self-equilibrated. We wish to find the closed-form elasticity solution for the stress, strain and displacement fields throughout the problem domain.
%
\begin{figure}[H]
\centering
\includegraphics[width=7cm]{img_src/wedge.pdf}
\caption{2D Self-equilibrated wedge.}
\label{fig:WEDGE}
\end{figure}

Under plane strain conditions the general 3D stress equilibrium equations (see \cref{eq:pde}) reduce to:
\begin{equation}
\begin{aligned}
&\pdv{\sigma_{xx}}{x}+\pdv{\tau_{xy}}{y}=0\\
&\pdv{\tau_{xy}}{x}+\pdv{\sigma_{yy}}{y}=0
\end{aligned}
\label{eq:equilibrium}
\end{equation}

while the kinematic relation (\cref{eq:kin}) reads

\begin{equation}
\begin{aligned}
\epsilon_{xx}&=\pdv{u}{x}\\
\epsilon_{yy}&=\pdv{v}{y}\\
\gamma_{xy}&=\pdv{u}{y} + \pdv{v}{x}
\end{aligned}
\label{eq:strain}
\end{equation}
where $u$ and $v$ are the horizontal and vertical displacements respectively.

\subsubsection*{Stress field}

The stress field can be obtained by simple inspection from the traction boundary conditions prescribed over the inclined surfaces yielding;

\begin{align*}
\sum F_x &= 0 \longrightarrow - \ell S\cos(\phi)  + \sigma_{xx}\ell \sin(\phi) = 0\\
\sum F_y &= 0 \longrightarrow - \ell S\sin(\phi) - \sigma_{yy}\ell \cos(\phi)=0
\end{align*}

and the following stress solution:

\begin{equation}
\begin{aligned}
\sigma_{xx}& = S \cot(\phi)\\
\sigma_{xx}& = -S\tan(\phi)\\
\tau_{xy}& = 0.
\end{aligned}
\label{eq:solution}
\end{equation}

In \cref{eq:solution} the condition $\tau_{xy}=0$ is due to symmetrie in the problem.

\subsubsection*{Traction boundary conditions}
Let us verify that the above stress solution satisfies the traction BC using the expression:

\[t_i^{\hat n} = \sigma _{ij} \hat n_{ij}.\]

Donoting the outward normals to the inclined surfaces of the wedge by $\hat{n}^1$,  $\hat{n}^2$, $\hat{n}^3$, $\hat{n}^4$ these are given by;
\begin{align*}
\hat{n}^1 &= -\sin(\phi)\hat{e}_{x}+\cos(\phi)\hat{e}_{y}\\
\hat{n}^2 &= -\sin(\phi)\hat{e}_{x}-\cos(\phi)\hat{e}_{y}\\
\hat{n}^3 &= +\sin(\phi)\hat{e}_{x}+\cos(\phi)\hat{e}_{y}\\
\hat{n}^4 &= +\sin(\phi)\hat{e}_{x}-\cos(\phi)\hat{e}_{y} \enspace
\end{align*}

where $\hat{e}_{x}$ and $\hat{e}_{y}$ are the reference unit vectors. Now, the components of the traction vector follow directly like

\[t_{i} = \sigma_{ij}\hat{n}_{j}\]

then over the face with normal $\hat{n}^1$ we have
\begin{align*}
t_{x} &= -S\cos(\phi)\\
t_{y} &= -S\sin(\phi)
\end{align*}
similarly, over the face with normal $\hat{n}^2$
\begin{align*}
t_{x} &= -S\cos(\phi)\\
t_{y} &= +S\sin(\phi)
\end{align*}
over the face with normal $\hat{n}^3$ 
\begin{align*}
t_{x} &= +S\cos(\phi)\\
t_{y} &= -S\sin(\phi)
\end{align*}
and finally, over the face with normal $\hat{n}^4$;
\begin{align*}
t_{x} &= +S\cos(\phi)\\
t_{y} &= +S\sin(\phi) \enspace .
\end{align*}

\subsubsection*{Strain field}
The strain field can be obtained after using the stress solution found in \cref{eq:solution} together with the constituive law given by \cref{eq:Hooke} which for a plane strain idalization takes the form:

\begin{equation}
\begin{aligned}
\epsilon_{xx}& = +\dfrac{S}{E}\left[\cot(\phi)+\nu \tan(\phi)\right] = +\dfrac{S}{E}K_{1}(\nu , \phi)\\
\epsilon_{yy}& = -\dfrac{S}{E}\left[\tan(\phi)+\nu \cot(\phi)\right] = -\dfrac{S}{E}K_{2}(\nu , \phi)\\
\gamma_{xy}& = 0.
\end{aligned}
\label{eq:strain part}
\end{equation}

\subsubsection*{Displacement field}
The displacement field is obtained after direct integration of the strains after using the fact that:

\[du_i=\epsilon_{ij}dx_j + \omega_{ij}dx_j\]

and the condition $\omega_{xy}=0$ also due to symmetrie , as follows:

\begin{align*}
u &= +\dfrac{S}{E} K_{1}(\nu , \phi)x + A\\
v &= -\dfrac{S}{E} K_{2}(\nu , \phi)y + B
\end{align*}

and where $A$ and $B$ are integration constants.

From the condition $u=0$ at $x=\ell\cos(\phi)$ we have that $A=-\dfrac{S}{E} K_{1}(\nu , \phi)\ell\cos(\phi)$ then it follows that
\[u=\dfrac{S}{E} K_{1}(\nu , \phi)(x-\ell\cos(\phi)).\]

Similarly, from the condition $v=0$ at $y=0$ we have that $B=0$ from which
\[v=-\dfrac{S}{E} K_{2}(\nu , \phi)y\]

\section{Boundary value problems}
%%
\subsection{Differential formulation}
%%
\subsection{Variational formulation}
%%
\todo{Add Variational formulation definition.}
According to Wikipedia \cite{wiki:variational_principle}

\begin{quotation}
A variational principle is a scientific principle used within the calculus of variations, which develops general methods for finding functions which minimize or maximize the value of quantities that depend upon those functions. For example, to answer this question: ``What is the shape of a chain suspended at both ends?" we can use the variational principle that the shape must minimize the gravitational potential energy.

According to Cornelius Lanczos, any physical law which can be expressed as a variational principle describes an expression which is self-adjoint. These expressions are also called Hermitian. Such an expression describes an invariant under a Hermitian transformation.
\end{quotation}

\subsubsection*{Principle of minimum potential energy}

\subsubsection*{Hu-Washizu Variational Principle}
Consider the following functional
\begin{equation}
\pi^* = \pi  - \intL_V \lambda_{ij}^\varepsilon (\varepsilon_{ij} - L_{ijk} u_k)\dd{V}  - \intL_{S_u} \lambda_i^u(u_i^{S_u} - \bar{ u}_i)\dd{S}
\label{eq:Hu}
\end{equation}
where
\begin{itemize}
\item $\pi$: is the potential energy functional.
\item $L_{ijk}$ is a differential operator such $\varepsilon_{ij} = L_{ijk} u_k$.
\item $S_u$ surface where essential boundary conditions are prescribed.
\item $\lambda_{ij}^\varepsilon $ and $\lambda_i^u$ are Lagrange multipliers.
\end{itemize}

We want to determine the so-called Euler equations resulting from the condition $\delta \pi^* = 0$. Applying the variational operator we have:
\begin{equation}
\begin{aligned}
\delta \pi *& = \delta \pi  - \intL_V \delta \lambda_{ij}^{\varepsilon}  (\varepsilon_{ij} - L_{ijk} u_k)\dd{V}- \intL_V \lambda_{ij}^\varepsilon (\delta \varepsilon_{ij} - L_{ijk}\delta u_k)\dd{V} \\
&-\intL_V \delta \lambda_i^u (u_i^{S_u} - \bar u_i)\dd{S} - \intL_V \lambda _i^u \delta u_i^{S_u} \dd{S}
\end{aligned}
\end{equation}

\begin{equation}
\begin{aligned}
\delta \pi * &= \intL_V C_{ijkl} \varepsilon_{kl} \delta  \varepsilon_{ij}\dd{V} - \intL_{S_t} t_i \delta{u_i} \dd{S}  - \intL_V f_i\delta {u_i}\dd{V} - \intL_V \lambda_{ij}^{\varepsilon} \delta \varepsilon_{ij}\dd{V}  + \intL_V \lambda_{ij}^{\varepsilon} L_{ijk}\delta {u_k}\dd{V}\\
&- \intL_V \delta \lambda_{ij}^{\varepsilon} (\varepsilon_{ij} - L_{ijk} u_k)\dd{V} - \intL_S \delta \lambda_i^u (u_i^{S_u} - \bar {u}_i)dS - \intL_{S_u} \lambda _i^u\delta {u_i}\dd{S} = 0
\end{aligned}
\end{equation}

using

\[\intL_V {(\lambda _{ij}^\varepsilon \delta {u_i}){,_j}dV = } \intL_V {\lambda _{ij}^\varepsilon \delta {u_{i,j}}dV}  + \intL_V {\lambda _{ij,j}^\varepsilon \delta {u_i}dV} \]

in the above we can write

\begin{align*}
\intL_V {\lambda _{ij}^\varepsilon {L_{ijk}}\delta {u_k}dV} & = \intL_V {{{(\lambda _{ij}^\varepsilon \delta {u_i})}_{,j}}dV}  - \intL_V {\lambda _{ij,j}^\varepsilon \delta {u_i}dV}\\
& = \intL_{{S_t}} {\lambda _{ij}^\varepsilon \delta {u_i}{{\hat n}_j}dS}  - \intL_V {\lambda _{ij,j}^\varepsilon \delta {u_i}dV}
\end{align*}

therefore


\begin{align*}
\delta \pi * &= \intL_V {({C_{ijkl}}{\varepsilon _{kl}} - \lambda _{ij}^\varepsilon )\delta {\varepsilon _{ij}}dV - \intL_{{S_t}} {{t_i}\delta {u_i}dS}  - \intL_V {{f_i}\delta {u_i}dV + } } \intL_{{S_t}} {\lambda _{ij}^\varepsilon \delta {u_i}{{\hat n}_j}dS}  - \intL_V {\lambda _{ij,j}^\varepsilon \delta {u_i}dV}\\
&- \intL_V {\delta \lambda _{ij}^\varepsilon ({\varepsilon _{ij}} - {L_{ijk}}{u_k})dV}  - \intL_{{S_u}} {\delta \lambda _i^u(u_i^{{S_u}} - {{\bar u}_i})dS}  - \intL_{{S_u}} {\lambda _i^u\delta {u_i}dS}  = 0
\end{align*}


\begin{align*}
&\intL_V {({C_{ijkl}}{\varepsilon _{kl}} - \lambda _{ij}^\varepsilon )\delta {\varepsilon _{ij}}dV
+ \intL_{{S_t}} {(\lambda _{ij}^\varepsilon {{\hat n}_j} - {t_i})\delta {u_i}dS}  - }\\
&\intL_V {(\lambda _{ij,j}^\varepsilon  + {f_i})\delta {u_i}dV}
- \intL_V {({\varepsilon _{ij}} - {L_{ijk}}{u_k})\delta \lambda _{ij}^\varepsilon dV}
- \intL_{{S_u}} {(u_i^{{S_u}} - {{\bar u}_i})\delta \lambda _i^udS}  - \cancel{\intL_{{S_u}} {\lambda _i^u\delta u_i^{{S_u}}dS} = } 0
\end{align*}


Now, imposing the conditions $\delta {\varepsilon _{ij}} \neq 0$, $\delta \lambda _{ij}^\varepsilon  \neq 0$, $\delta {u_i} \neq 0$ in $S_t$, $\delta {u_i} \neq 0$ in $V$ and $\delta \lambda _i^u \neq 0$ in $S_u$ we have
%
\begin{align}
&\lambda _{ij}^\varepsilon  = C_{ijkl} \varepsilon_{kl}\\
&\varepsilon_{ij} = {L_{ijk}}{u_k}\\
t_i &= \lambda_{ij}^\varepsilon \hat{n}_j\\
&\lambda _{ij,j}^\varepsilon  + {f_i} = 0\\
&u_i^{S_u} = \bar{u}_i
\end{align}

\subsection{Weighted residual methods}
\subsubsection*{Application to the Navier equations}
\subsubsection*{Application to the acoustic wave equation}

\section[Discretization of the PVW using FEM]{Discretization of the PVW via the FEM}
\subsection*{A simple spring-mass system}
The simple problem of a spring-mass system considered next resembles most of the algorithmic aspects of a finite element code with the advantage that the problem is already a discrete mechanical system. The problem consists of an assamblage of masses joined by different springs submitted to time varying loads. Each spring plays the role of a finite element and each mass is analogous to a nodal point in a finite element algorithm. For instance the full system may be like the one shown in \cref{fig:bathe}:


\begin{figure}[H]
\centering
\includegraphics[width=10cm]{img_src/system.pdf}
\caption{Typical assemblage of springs and masses.}
\label{fig:bathe}
\end{figure}


Consider a typical spring (finite element) like the one shown in \cref{fig:springel}

\begin{figure}[H]
\centering
\includegraphics[width=10cm]{img_src/springel.pdf}
\caption{Typical spring element.}
\label{fig:springel}
\end{figure}

The relation between the force and the relative displacement can be written like;

\[{f_1} = K({u_1} - {u_2})\]

and from equilibrium we have;

\[{f_1} + {f_2} = 0\]

which yields the following force-displacement relationship for a tyical spring element:

\begin{equation}
\left\{ {\begin{array}{*{20}{c}}
{{f_1}}\\
{{f_2}}
\end{array}} \right\} = K\left[ {\begin{array}{*{20}{c}}
{1.0}&{ - 1.0}\\
{ - 1.0}&{1.0}
\end{array}} \right]\left\{ {\begin{array}{*{20}{c}}
{{u_1}}\\
{{u_2}}
\end{array}} \right\}
\label{Kspring}
\end{equation}

On the other hand, the equilibrium equation for a typical mass with displacement $u_j$  (see \cref{fig:dclmass}) and attached to springs $i$ and $i+1$ read;

\begin{equation}
f_2^i + f_1^{i + 1} + {m_j}\frac{{d{V_j}}}{{dt}} = {P_j}.
\label{equilmass}
\end{equation}

which can be written in terms of displacements using \cref{Kspring} like;

\[({K^i} + {K^{i + 1}}){u_j} - {K^i}{u_{j - 1}} - {K^{i + 1}}{u_{j + 1}} + {m_j}\frac{{d{V_j}}}{{dt}} = {P_j}.\]


\begin{figure}[H]
\centering
\includegraphics[width=8cm]{img_src/dclmass.pdf}
\caption{Free body diagram for a typical mass connceted to springs $i$ and $i+1$.}
\label{fig:dclmass}
\end{figure}

Considering now the complete system of masses and springs leads to a system of linear equations of the form;

\begin{equation}
\left[ {{K_G}} \right]\left\{ {{U_G}} \right\} + \left[ M \right]\left\{ {{A_G}} \right\} = \left\{ {{F_G}} \right\}.
\label{global}
\end{equation}

where each equation represents the equilibrium of a given mass. The system given by \cref{global} can be solved in the displacements $U_G$. The pseudo-code shown in \cref{springsalg} presents all the steps required to sove the problem in the context of the finite element method. In that code the so-called DME operator is an equation assembly array indicating how each element contributes to the global stiffness and mass matrix.


\begin{algorithm}[H]
 \SetAlgoLined
 \KwData{Problem paramters; NUMNP, NUMEL, NMATP}
 \KwResult{Displacements and spring forces}
 Create $DM$E operator\;
 Assemble $K^G$, $F^G$\;
\While{$j \leq 1, NUMEL$}{
\[
\begin{aligned}
K^G \leftarrow K^G+K^i\\
F^G \leftarrow F^G+F^i\\
\end{aligned}
\]
}
Impose BCs\;
Solve $[K^G]U=F^G$\\
Find internal forces
\caption{Springs Algorithm}
\label{springsalg}
\end{algorithm}


\subsection{Basic elements of interpolation theory}

\subsection{Formulation of the finite element matrices}
We now discretize the principle of virtual work repeated below for completeness:

\begin{equation} \label{pvw_2}
\intL_V \sigma_{ij} \delta u_{i,j} dV - \intL_V f_i \delta u_i dV - \intL_{S_t} t_i^n \delta u_i dS = 0.
\end{equation}

For that purpose we will divide the complete domain $V$ into $N$-finite non-overlapping subdomains over each one of which we will approximate the solution in terms of local interpolating functions. Since the PVW (or weak form of the BVP) has been casted into an integral representation, it is possible to build the total integral considering the contribution of the $N$-sub-domains as follows;

\begin{equation}\label{pvw_dis}
\sum_{e=1}^{NEL} \intL_{V^e} \sigma_{ij} \delta u_{i,j} d{V^e} - \intL_V f_i \delta u_i d{V^e} - \intL_{S_t} t_i^n \delta u_id{S^e} = 0 
\end{equation}

For easiness consider a single subdomain
\begin{equation} \label{pvw_sing}
\intL_V \sigma_{ij} \delta u_{i,j} dV - \intL_V f_i\delta u_idV - \intL_{S_t} t_i^n \delta u_i dS = 0.
\end{equation}

The involved functions (e.g., displacements, strain, stresses) will be approximated via interpolation of the solution over a determined number of points termed in what follows nodes. Assume for instance that over element $e$ containing $n$ such nodes we know the displacements vector $u_i$. Furthermore, let the displacements for the $p$-node $u^P=[u^P, v^P, w^P]$. Using ideas from interpolation theory it is now possible to approximate the displacements vector over an arbitrary point $\vb{x}$ inside the element as follows
\[u_i(\vb x) = N_i^1(\vb x)u^1 + N_i^2(\vb x)u^2 + \cdots + N_i^P(\vb x)u^P + \cdots + N_i^n(\vb x)u^n\]
or in more general form
\begin{equation} \label{bas_interpol}
{u_i}(\vb x) = N_i^Q(\vb x){u^Q}
\end{equation}

and where the caption superscripts indicate summation over the number of nodes of the element while the subscript refers to the physical character of the variable being interpolated.

\begin{align*}
&\varepsilon_{ij}(\vb x) = B_{ij}^Q(\vb x){u^Q}\\
&\varepsilon_{ij}(\vb x) = \frac{1}{2}( u_{i,j} + u_{j,i} )\\
&\varepsilon_{ij}(\vb x) = \frac{1}{2}\left(\pdv{N_i^Q}{x_j} + \pdv{N_j^Q}{x_i} \right){u^Q}\\
&B_{ij}^Q = \frac{1}{2}\left(\pdv{N_i^Q}{x_j} + \pdv{N_j^Q}{x_i} \right)\\
&\delta {u_i} = N_i^Q(\vb x)\delta {u^Q}\\
&\intL_V C_{ijkl} B_{kl}^P u^P B_{ij}^Q\delta u^Q \dd{V} - \intL_V f_i N_i^Q\delta {u^Q}\dd{V}  - \intL_{S_t} t_i^n N_i^Q\delta {u^Q} \dd{S} = 0\\
&\delta {u^Q}\intL_V B_{ij}^Q C_{ijkl} B_{kl}^P\dd{V}{u^P} - \delta {u^Q}\intL_V N_i^Q{f_i}\dd{V}  - \delta {u^Q}\intL_{S_t} N_i^Qt_i^n\dd{S} = 0 \\
&\delta u^Q f_\sigma ^Q - \delta {u^Q}f_V^Q - \delta u^Q f_c^Q = 0\\
&f_\sigma ^Q - f_V^Q - f_c^Q = 0\\
&f_\sigma ^Q = \intL_V B_{ij}^Q C_{ijkl} B_{kl}^P\dd{V} u^P \equiv K^{QP} u^P\\
&f_V^Q = \intL_V N_i^Q f_i\dd{V} \\
&f_c^Q = \intL_{S_t} N_i^Qt_i^n \dd{S} \\
&K^{QP} u^P = f_V^Q + f_c^Q
\end{align*}

\subsubsection{Formulation in the physical space}
\subsubsection{Formulation in the natural space: the continuum mechanics analogy}
In typical finite element equilibrium equations we need to perform integration over the reference element domain $V_0(\vb{x})$ corresponding to originally arbitrarily shaped sub-domains as created during the meshing process.  In order to proceed with this integration it is useful to consider the following continuum mechanics analogy.

First assume that the actual physical domain $V_0(\vb{x})$ is the result of a deformation process imparted upon the natural domain as shown in \cref{fig:natural domain}. In this analogy, the physical domain $V_0(\vb{x})$ is regarded like a ``deformed'' configuration at an imaginary time $t=t$, while the natural ``undeformed'' domain $V(\vb{r})$   is treated like a reference undeformed configurations at time $t=0$. Both configurations are assumed to be connected through a deformation process;


\begin{equation}
\begin{aligned}
\vb{X}&=\vb{X}(\vb{r})\\
\vb{r}&=\vb{r}(\vb{X})
\end{aligned}
\label{eq:motion}
\end{equation}

\begin{figure}[h]
\centering
\includegraphics[width=8cm]{img/figure1.pdf}
\caption{Definition of the natural domain}
\label{fig:natural domain}
\end{figure}

 

In \cref{eq:motion} we can understand $\vb{r}$ like a material (Lagrangian) variable and $\vb{X}$ like a spatial (or Eulerian) variable. Using the continuum mechanics analogy it is clear that the ``deformation'' process at the continuum level is fully characterized by the ``deformation'' gradient or Jacobian of the transformation \cref{eq:motion} and defined according to;

\begin{equation}
dX_i=\dfrac{\partial X_i}{\partial r_J}dr_J\equiv J_{iJ}dr_{J}
\label{eq:gradient}
\end{equation}

where $dr_{J}$ and $dX_i$ represent material vectors in the original and deformed configuration. From \cref{eq:gradient} it is evident that the Jacobian contains all the information describing the change of the physical sub-domain with respect to the natural element. For the element integration process we will assume that every element $V(\vb{r})$ in the natural domain deforms into the physical element $V_0(\vb{X})$, thus allowing us to write typical terms like the ones in the material stiffness matrix
\begin{equation}
\intL_{V(\vb{X})} \hat{B}_{ij}^K(\vb{X}) C_{ijkl} \hat{B}_{kl}^P(\vb{X}) dV(\vb{X})\equiv \intL_{V_0(\vb{r})} \hat{B}_{ij}^K(\vb{r}) C_{ijkl} \hat{B}_{kl}^P(\vb{r})J dV_0(\vb{r})
\label{eq:matmatrix}
\end{equation}
where we have used $dV(\vb{X})=JdV(\vb{r})$, with $J$ being the determinant of the deformation gradient and in general we transform functions between the natural and physical space making use of \cref{eq:motion} according to
\begin{equation}
f(\vb{r})=F[\vb{X}(\vb{r})]
\label{eq:funtrans}
\end{equation}

	 								
\subsubsection*{Interpolation scheme}
Having identified the fact that the integration process will take place in the natural domain, we will approach the interpolation process directly in this natural space. In the case of the displacement based finite element method all the involved variables will then be obtained via interpolation of nodal displacements. For instance, assume that a given problem variable is defined in the physical space by the tensor $\Phi_{ik...p}(\vb{X})$. The interpolated variable is then written like;

\begin{equation}
\Phi_{ij...p}(\vb{X})=H_{ij...p}^K(\vb{r})\hat{u}^K
\label{eq:interpol}
\end{equation}	 						

where $\hat{u}^K$ represents a vector of nodal points displacements, see \cref{fig:interpol nat dom}, and $H_{ij...p}^K(\vb{r})$ is an interpolator which keeps the tensorial character of the original physical variable $\Phi_{ik...p}(\vb{X})$ and where the super-index makes reference to a nodal identifier (with the summation convention in place).


\begin{figure}[h]
\centering
\includegraphics[width=4cm]{img/figure2.pdf}
\caption{General interpolation strategy in the natural domain}
\label{fig:interpol nat dom}
\end{figure}
 


Since the primary variable corresponds to displacements it must be kept in mind that $H_{ij...p}^K(\vb{r})$ corresponds to combinations of derivatives (or other arbitrary combinations) of the basic element shape functions defined in;


\begin{equation}
u_i(\vb{X})=N_i^K(\vb{r})\hat{u}^K
\label{eq:el interpol}
\end{equation}



For the general interpolation process we need two kinds of transformations.  First we need to transform integrals over the physical space into integrals into the natural space which corresponds to
\begin{equation}
\intL_{V(\vb{X})} F(\vb{X})dV(\vb{X})\equiv \intL_{V_0(\vb{r})} f(\vb{r})J dV_0(\vb{r})
\label{gen trans}
\end{equation}



Second we need to relate spatial differentiation in both, the physical and spatial domains.  Let us define these operators like $\nabla_i^X$ and $\nabla_I^r$ respectively. It then follows from \cref{eq:funtrans} that
\begin{equation}
\dfrac{\partial F}{\partial X_i}=\dfrac{\partial f}{\partial r_J}\dfrac{\partial r_J}{\partial X_i}
\label{eq:chain}
\end{equation}
from where we can establish the connection between the two operators like


\begin{equation}
\nabla_i^X=J_{iJ}^{-1}\nabla_J^r
\label{eq:fundamental}
\end{equation}


\subsubsection*{The fundamental interpolator}
We further define the fundamental interpolator giving rise to gradients of the primary displacement variable in the physical space according to
\begin{equation}
u_{i,j}(\vb{X})=L_{ij}^K(\vb{r})\hat{u}^K
\label{eq:fund operator}
\end{equation}


This fundamental interpolator  $L_{ik}^K(\vb{r})$ is derived after using \cref{eq:el interpol} and \cref{eq:fundamental} in the physical displacement gradient definition as shown next
\begin{align*}
u_{i,j}(\vb{X})&=\nabla_j^X u_i(\vb{X})\\
u_{i,j}(\vb{X})&=\nabla_j^X N_i^K(\vb{r})\hat{u}^K\\
u_{i,j}(\vb{X})&=J_{jQ}^{-1}\nabla_Q^r N_i^K(\vb{r})\hat{u}^K\\
u_{i,j}(\vb{X})&=J_{jQ}^{-1}N_{i,Q}^K(\vb{r})\hat{u}^K
\end{align*}
then
\begin{equation}
L_{ij}^K(\vb{r})=J_{jQ}^{-1}N_{i,Q}^K(\vb{r})
\label{eq:fundamental interpolator}
\end{equation}

\subsubsection*{Elemental stiffness matrix}
The elemental material stiffness matrix computed in the natural domain of \cref{fig:Nat domain} reads;

\begin{equation}
K^{KP}=\intL_{V_0(\vb{r})} \hat{B}_{ij}^K(\vb{r}) C_{ijkl} \hat{B}_{kl}^P(\vb{r})J dV_0(\vb{r})\equiv \intL_{r=-1}^{r=+1}\intL_{s=-1}^{s=+1} \hat{B}_{ij}^K(r,s) C_{ijkl} \hat{B}_{kl}^P(r,s)J(r,s) \mathrm{d}r\mathrm{d}s
\label{eq:elematrix}
\end{equation}



\begin{figure}[h]
\centering
\includegraphics[width=6cm]{img/figure3.pdf}
\caption{Natural domain of integration}
\label{fig:Nat domain}
\end{figure}	 		
 

Once the interpolator $\hat{B}_{ij}^K(\vb{r})$ has been identified the elemental stiffness matrix is obtained via numerical integration (quadrature) as described in \eqref{eq:eleinetgration};

\begin{equation}
\intL_{r=-1}^{r=+1}\intL_{s=-1}^{s=+1} \hat{B}_{ij}^K(r,s) C_{ijkl} \hat{B}_{kl}^P(r,s)J(r,s) \mathrm{d}r\mathrm{d}s\approx \sum_{i,j=1}^\text{NGPTS} \alpha_i \alpha_j \hat{B}_{kl}^K(r_i,s_j)C_{ijkl} \hat{B}_{kl}^P(r_i,s_j) J(r_i,s_j)
\label{eq:eleintegration}
\end{equation}

	 											
and where NGPTS corresponds to the number of integration points, $\alpha_j$ is a weighting factor and $r_i,s_j$   are the coordinates of a typical point $\vb{r}$ in the natural space of \cref{fig:Nat domain}.

 
\begin{figure}[h]
\centering
\includegraphics[width=6cm]{img/figure4.pdf}
\caption{Natural integration domain showing quadrature evaluation nodes}
\label{fig:integration domain}
\end{figure}	 


One important aspect of the numerical integration that has to be kept in mind is accuracy.  Depending on the particularly selected integration scheme, the number of introduced integration points fixes the maximum polynomial order of the considered functions that can be integrated accurately.  In the case of the integrand in \cref{eq:eleintegration}, it is clear that this order increases as the distortion of the physical element  with respect to the natural element increases.  One way of dealing with this dependency of accuracy with element distortion is to make use of adaptative integration techniques which are numerically expensive.  What is actually done in standard FEM analysis is to choose the number of quadrature points beforehand and introduce distortion related error criteria inside the code in such a way that some sort of validation is performed before the numerical integration process is started.

\subsubsection*{Strain displacement interpolator for the infinitesimal strain tensor}
The $Q$-th nodal contribution to the infinitesimal strain-displacement interpolator can be obtained in explicit form as follows. Let $L_x^Q$ and $L_y^Q$ be the spatial differential operators in $x$ and $y$ respectively. We have after expanding \cref{eq:fundamental interpolator}
%
\begin{align*}
L_x^Q & = J_{xP}^{-1}\frac{\partial N^Q}{\partial r_P} \equiv J_{xr}^{-1}\frac{\partial N^Q}{\partial r} + J_{xs}^{-1}\frac{\partial N^Q}{\partial s}\\
L_y^Q & = J_{yP}^{-1}\frac{\partial N^Q}{\partial r_P} \equiv J_{yr}^{-1}\frac{\partial N^Q}{\partial r} + J_{ys}^{-1}\frac{\partial N^Q}{\partial s}
\end{align*}
%
or in matrix form
%
\begin{equation}
\begin{Bmatrix}
L_x^Q\\
L_y^Q
\end{Bmatrix} = 
\begin{bmatrix}
J_{xP}^{-1} &J_{xs}^{- 1}\\
J_{yr}^{-1} &J_{ys}^{- 1}
\end{bmatrix}
\begin{Bmatrix}
\frac{\partial N^Q}{\partial r}\\
\frac{\partial N^Q}{\partial s}
\end{Bmatrix}
\end{equation}

The $Q$-th nodal contribution is then assembled as follows;


\begin{equation}
\begin{Bmatrix}
\pdv{u}{x}\\
\pdv{v}{y}\\
\pdv{u}{y} + \pdv{v}{x}
\end{Bmatrix} =
\begin{bmatrix}
 &L_x^Q &0 \\
\cdots &0 &L_y^Q &\cdots\\
 &L_y^Q &L_{xy}^Q
\end{bmatrix}
\begin{Bmatrix}
\vdots\\
u^Q\\
v^Q\\
\vdots
\end{Bmatrix}
\label{eq:strain inter}
\end{equation}

\begin{algorithm}[H]
\SetAlgoLined
\KwData{Nodal coordinates $x^Q$}
\KwResult{Strain-displacement interpolator $B_{ij}^Q$ }
Compute Jacobian ${J_{iJ}} = \pdv{N_i^Q}{r_J}{\hat x}^Q$\\
Invert Jacobian  ${J_{iJ}} \to J_{iJ}^{ - 1}$\\
Compute fundamental interpolator $L_{ij}^Q = J_{jP}^{ - 1}\pdv{N_i^Q}{r_P}$\\
Assemble $B_{ij}^Q = \frac{1}{2}\left( {L_{ij}^Q + L_{ji}^Q} \right)$ 
\caption{Strain-displacement interpolator}
\end{algorithm}


